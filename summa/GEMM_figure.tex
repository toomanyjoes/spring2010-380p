\begin{figure}[h]

\resetsteps      % Reset all the commands to create a blank worksheet  

% Define the operation to be computed

\renewcommand{\routinenameL}{ C = AB + \hat{C} \mbox{~~~~~(Variant 1, Unblocked)}}

% Step 3: Loop-guard 

\renewcommand{\guard}{
  n( A_L ) < n( A )
}

% Step 4: Define Initialize 

\renewcommand{\partitioningsL}{
  $
  A \rightarrow
  \FlaOneByTwo{A_L}{A_R}
  $
,
  $
  B \rightarrow
  \FlaTwoByOne{B_{T}}
              {B_{B}}
  $
}

\renewcommand{\partitionsizesL}{
$ A_L $ has $ 0 $ columns,
$ B_{T} $ has $ 0 $ rows
}

% Step 5a: Repartition the operands 

\renewcommand{\repartitioningsL}{
$
  \FlaOneByTwo{A_L}{A_R}
\rightarrow  \FlaOneByThreeR{A_0}{a_1}{A_2}
$
,
\\
$
  \FlaTwoByOne{ B_T }
              { B_B }
\rightarrow
  \FlaThreeByOneB{B_0}
                 {b_1^T}
                 {B_2}
$
}

\renewcommand{\repartitionsizesL}{
$ a_1 $ has $1$ column,
$ b_1 $ has $ 1 $ row}

% Step 5b: Move the double lines 

\renewcommand{\moveboundariesL}{
$
  \FlaOneByTwo{A_L}{A_R}
\leftarrow  \FlaOneByThreeL{A_0}{a_1}{A_2}
$
,
\\
$
  \FlaTwoByOne{ B_T }
              { B_B }
\leftarrow
  \FlaThreeByOneT{B_0}
                 {b_1^T}
                 {B_2}
$
}

% Step 8: Insert the updates required to change the 
%         state from that given in Step 6 to that given in Step 7
% Note: The below needs editing!!!

\renewcommand{\updateL}{
$
  \begin{array}{l}
    C = a_1 b^T_1 + C
  \end{array}
$
}
%%%%%%%%%%%%%%%%%%%%%%%%%%%%%%%%


% Define the operation to be computed

\renewcommand{\routinenameR}{ C = AB + \hat{C} \mbox{~~~~~(Variant 1, Blocked)} }

% Step 3: Loop-guard 

\renewcommand{\guard}{
  n( A_L ) < n( A )
}

% Step 4: Define Initialize 

\renewcommand{\partitioningsR}{
  $
  A \rightarrow
  \FlaOneByTwo{A_L}{A_R}
  $
,
  $
  B \rightarrow
  \FlaTwoByOne{B_{T}}
              {B_{B}}
  $
}

\renewcommand{\partitionsizesR}{
$ A_L $ has $ 0 $ columns,
$ B_{T} $ has $ 0 $ rows
}

% Step 5a: Repartition the operands 

\renewcommand{\blocksize}{b}

\renewcommand{\repartitioningsR}{
$
  \FlaOneByTwo{A_L}{A_R}
\rightarrow  \FlaOneByThreeR{A_0}{A_1}{A_2}
$
,
\\
$
  \FlaTwoByOne{ B_T }
              { B_B }
\rightarrow
  \FlaThreeByOneB{B_0}
                 {B_1}
                 {B_2}
$
}

\renewcommand{\repartitionsizesR}{
$ A_1 $ has $b$ columns,
$ B_1 $ has $ b $ rows}

% Step 5b: Move the double lines 

\renewcommand{\moveboundariesR}{
$
  \FlaOneByTwo{A_L}{A_R}
\leftarrow  \FlaOneByThreeL{A_0}{A_1}{A_2}
$
,
\\
$
  \FlaTwoByOne{ B_T }
              { B_B }
\leftarrow
  \FlaThreeByOneT{B_0}
                 {B_1}
                 {B_2}
$
}

% Step 8: Insert the updates required to change the 
%         state from that given in Step 6 to that given in Step 7
% Note: The below needs editing!!!

\renewcommand{\updateR}{
$
  \begin{array}{l}
    C = A_1 B_1 + C
  \end{array}
$
}

\FlaAlgorithmTwo
 
\end{figure} 
