\begin{figure}[tbp]

\resetsteps      % Reset all the commands to create a blank worksheet  

% Define the operation to be computed

\renewcommand{\routinenameL}{ C = AB + \hat{C} \mbox{~~~~~(Variant 3, Unblocked)}}

% Step 3: Loop-guard 

\renewcommand{\guard}{
  m( C_T ) < m( C )
}

% Step 4: Define Initialize 

\renewcommand{\partitioningsL}{
  $
  C \rightarrow
  \FlaTwoByOne{C_{T}}
              {C_{B}}
  $
,
  $
  A \rightarrow
  \FlaTwoByOne{A_{T}}
              {A_{B}}
  $
}

\renewcommand{\partitionsizesL}{
$ C_{T} $ has $ 0 $ rows,
$ A_{T} $ has $ 0 $ rows
}

% Step 5a: Repartition the operands 

\renewcommand{\repartitioningsL}{
$
  \FlaTwoByOne{ C_T }
              { C_B }
\rightarrow
  \FlaThreeByOneB{C_0}
                 {c_1^T}
                 {C_2}
$
,
$
  \FlaTwoByOne{ A_T }
              { A_B }
\rightarrow
  \FlaThreeByOneB{A_0}
                 {a_1^T}
                 {A_2}
$
}

\renewcommand{\repartitionsizesL}{
$ c_1 $ has $ 1 $ row,
$ a_1 $ has $ 1 $ row}

% Step 5b: Move the double lines 

\renewcommand{\moveboundariesL}{
$
  \FlaTwoByOne{ C_T }
              { C_B }
\leftarrow
  \FlaThreeByOneT{C_0}
                 {c_1^T}
                 {C_2}
$
,
$
  \FlaTwoByOne{ A_T }
              { A_B }
\leftarrow
  \FlaThreeByOneT{A_0}
                 {a_1^T}
                 {A_2}
$
}

% Step 8: Insert the updates required to change the 
%         state from that given in Step 6 to that given in Step 7
% Note: The below needs editing!!!

\renewcommand{\updateL}{
$
  \begin{array}{l}
    c_1^T = a_1^T B + c_1^T
  \end{array}
$
}

%%%%%%%%%%%%%%%%%%%%%%%%%

% Define the operation to be computed

\renewcommand{\routinenameR}{ C = AB + \hat{C} \mbox{~~~~~(Variant 3, Blocked)}}

% Step 3: Loop-guard 

\renewcommand{\guard}{
  m( C_T ) < m( C )
}

% Step 4: Define Initialize 

\renewcommand{\partitioningsR}{
  $
  C \rightarrow
  \FlaTwoByOne{C_{T}}
              {C_{B}}
  $
,
  $
  A \rightarrow
  \FlaTwoByOne{A_{T}}
              {A_{B}}
  $
}

\renewcommand{\partitionsizesR}{
$ C_{T} $ has $ 0 $ rows,
$ A_{T} $ has $ 0 $ rows
}

% Step 5a: Repartition the operands 

\renewcommand{\blocksize}{b}

\renewcommand{\repartitioningsR}{
$
  \FlaTwoByOne{ C_T }
              { C_B }
\rightarrow
  \FlaThreeByOneB{C_0}
                 {C_1}
                 {C_2}
$
,
$
  \FlaTwoByOne{ A_T }
              { A_B }
\rightarrow
  \FlaThreeByOneB{A_0}
                 {A_1}
                 {A_2}
$
}

\renewcommand{\repartitionsizesR}{
$ C_1 $ has $ b $ rows,
$ A_1 $ has $ b $ rows}

% Step 5b: Move the double lines 

\renewcommand{\moveboundariesR}{
$
  \FlaTwoByOne{ C_T }
              { C_B }
\leftarrow
  \FlaThreeByOneT{C_0}
                 {C_1}
                 {C_2}
$
,
$
  \FlaTwoByOne{ A_T }
              { A_B }
\leftarrow
  \FlaThreeByOneT{A_0}
                 {A_1}
                 {A_2}
$
}

% Step 8: Insert the updates required to change the 
%         state from that given in Step 6 to that given in Step 7
% Note: The below needs editing!!!

\renewcommand{\updateR}{
$
  \begin{array}{l}
    C_1 = A_1 B + C_1
  \end{array}
$
}

\FlaAlgorithmTwo
 
\end{figure} 
